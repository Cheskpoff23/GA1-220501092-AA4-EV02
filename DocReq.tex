\documentclass{article}
%%%%%%%%%%%%%%%%%%%%%%%%%%%%% Define Article %%%%%%%%%%%%%%%%%%%%%%%%%%%%%%%%%%
%%%%%%%%%%%%%%%%%%%%%%%%%%%%%%%%%%%%%%%%%%%%%%%%%%%%%%%%%%%%%%%%%%%%%%%%%%%%%%%

%%%%%%%%%%%%%%%%%%%%%%%%%%%%% Using Packages %%%%%%%%%%%%%%%%%%%%%%%%%%%%%%%%%%
\usepackage{geometry}
\usepackage{graphicx}
\usepackage{amssymb}
\usepackage{amsmath}
\usepackage{amsthm}
\usepackage{empheq}
\usepackage{mdframed}
\usepackage{booktabs}
\usepackage{lipsum}
\usepackage{graphicx}
\usepackage{color}
\usepackage{psfrag}
\usepackage{pgfplots}
\usepackage{bm}
%%%%%%%%%%%%%%%%%%%%%%%%%%%%%%%%%%%%%%%%%%%%%%%%%%%%%%%%%%%%%%%%%%%%%%%%%%%%%%%

% Other Settings

%%%%%%%%%%%%%%%%%%%%%%%%%% Page Setting %%%%%%%%%%%%%%%%%%%%%%%%%%%%%%%%%%%%%%%
\geometry{letterpaper, margin=2.54cm}

%%%%%%%%%%%%%%%%%%%%%%%%%% Define some useful colors %%%%%%%%%%%%%%%%%%%%%%%%%%
\definecolor{ocre}{RGB}{243,102,25}
\definecolor{mygray}{RGB}{243,243,244}
\definecolor{deepGreen}{RGB}{26,111,0}
\definecolor{shallowGreen}{RGB}{235,255,255}
\definecolor{deepBlue}{RGB}{61,124,222}
\definecolor{shallowBlue}{RGB}{235,249,255}
%%%%%%%%%%%%%%%%%%%%%%%%%%%%%%%%%%%%%%%%%%%%%%%%%%%%%%%%%%%%%%%%%%%%%%%%%%%%%%%

%%%%%%%%%%%%%%%%%%%%%%%%%% Define an orangebox command %%%%%%%%%%%%%%%%%%%%%%%%
\newcommand\orangebox[1]{\fcolorbox{ocre}{mygray}{\hspace{1em}#1\hspace{1em}}}
%%%%%%%%%%%%%%%%%%%%%%%%%%%%%%%%%%%%%%%%%%%%%%%%%%%%%%%%%%%%%%%%%%%%%%%%%%%%%%%

%%%%%%%%%%%%%%%%%%%%%%%%%%%% English Environments %%%%%%%%%%%%%%%%%%%%%%%%%%%%%
\newtheoremstyle{mytheoremstyle}{3pt}{3pt}{\normalfont}{0cm}{\rmfamily\bfseries}{}{1em}{{\color{black}\thmname{#1}~\thmnumber{#2}}\thmnote{\,--\,#3}}
\newtheoremstyle{myproblemstyle}{3pt}{3pt}{\normalfont}{0cm}{\rmfamily\bfseries}{}{1em}{{\color{black}\thmname{#1}~\thmnumber{#2}}\thmnote{\,--\,#3}}
\theoremstyle{mytheoremstyle}
\newmdtheoremenv[linewidth=1pt,backgroundcolor=shallowGreen,linecolor=deepGreen,leftmargin=0pt,innerleftmargin=20pt,innerrightmargin=20pt,]{theorem}{Theorem}[section]
\theoremstyle{mytheoremstyle}
\newmdtheoremenv[linewidth=1pt,backgroundcolor=shallowBlue,linecolor=deepBlue,leftmargin=0pt,innerleftmargin=20pt,innerrightmargin=20pt,]{definition}{Definition}[section]
\theoremstyle{myproblemstyle}
\newmdtheoremenv[linecolor=black,leftmargin=0pt,innerleftmargin=10pt,innerrightmargin=10pt,]{problem}{Problem}[section]
%%%%%%%%%%%%%%%%%%%%%%%%%%%%%%%%%%%%%%%%%%%%%%%%%%%%%%%%%%%%%%%%%%%%%%%%%%%%%%%

%%%%%%%%%%%%%%%%%%%%%%%%%%%%%%% Plotting Settings %%%%%%%%%%%%%%%%%%%%%%%%%%%%%
\usepgfplotslibrary{colorbrewer}
\pgfplotsset{width=8cm,compat=1.9}
%%%%%%%%%%%%%%%%%%%%%%%%%%%%%%%%%%%%%%%%%%%%%%%%%%%%%%%%%%%%%%%%%%%%%%%%%%%%%%%

%%%%%%%%%%%%%%%%%%%%%%%%%%%%%%% Title & Author %%%%%%%%%%%%%%%%%%%%%%%%%%%%%%%%
\author{Gustavo Vergara}
%%%%%%%%%%%%%%%%%%%%%%%%%%%%%%%%%%%%%%%%%%%%%%%%%%%%%%%%%%%%%%%%%%%%%%%%%%%%%%%


\begin{document}

\begin{titlepage}
    \centering
    
    
    \vspace{3cm}
    {\scshape\large DOCUMENTO CON ESPECIFICACIÓN DE REQUERIMIENTOS DEL SISTEMA DE GESTION DE CITAS MEDICAS PARA LA CLINICA REGIONAL DE MONTELIBANO \par}
    \vspace{7cm}
    \textbf\large\scshape{\par}
         \vspace{0.5cm}
         
    {\Large Vergara Pareja Gustavo\par}
    \vspace{7cm}
    {\scshape\Large Tecnologia en Analisis y Desarrollo de Software \par}
    \vspace{1cm}
    {\scshape\Large SENA - Centro Agropecuario Regional Cauca\par}
    \vspace{1cm}
    {\Large \today \par}
    \end{titlepage}

\tableofcontents

\newpage

\begin{flushleft}
    \large \textbf{EVIDENCIA A SOLUCIONAR}\\
    \vspace{1cm}
    
    \large Evidencia de producto: GA1-220501092-AA4-EV02 documento con especificación de requerimientos
    Respecto a lista de requerimientos el aprendiz deberá agregar una sección donde se describa cada requisito usando los siguientes elementos del estándar IEEE830.
    \begin{itemize}
    \item Perspectiva del producto.
    \item Funciones del producto.
    \item Características de los usuarios.
    \item Restricciones.
    \item Requisitos funcionales (formato de casos de uso).
    \item Requisitos no funcionales.
    \end{itemize}
    Respecto a la lista de requerimientos el aprendiz deberá agregar una sección donde se describa cada requisito usando la estructura de historias de usuario con los siguientes elementos por historia:
    \begin{itemize}
    \item Número de historia (priorizada).
    \item Nombre de la historia.
    \item Usuario.
    \item Puntos estimados de esfuerzo.
    \item Descripción de la historia de usuario.
    \item Observaciones.
    \item Criterios de aceptación.
    \end{itemize}
    \end{flushleft}
    \newpage


\section{Introducción}
\large El software de gestión de citas para una clínica es una herramienta diseñada para facilitar la programación y administración de citas médicas. Este tipo de software permite a los pacientes solicitar citas, a los médicos y personal de la clínica programar y gestionar las citas, y a los administradores supervisar y optimizar el flujo de pacientes. El objetivo principal de este software es mejorar la eficiencia y la experiencia del paciente en la clínica.

\newpage

\section{Requisitos Funcionales}
\subsection{Registro de pacientes}
El sistema debe permitir el registro de nuevos pacientes, recopilando información como nombre, fecha de nacimiento, información de contacto y antecedentes médicos relevantes.

\subsection{Programación de citas}
El sistema debe permitir a los médicos y al personal de la clínica programar citas para los pacientes, teniendo en cuenta la disponibilidad de los médicos y las salas de consulta.

\subsection{Recordatorios de citas}
El sistema debe enviar recordatorios automáticos de citas a los pacientes a través de mensajes de texto, correos electrónicos u otras formas de comunicación.

\subsection{Gestión de cancelaciones y reprogramaciones}
El sistema debe permitir a los pacientes cancelar o reprogramar citas, y notificar automáticamente a los médicos y al personal de la clínica sobre los cambios.

\subsection{Historial de citas}
El sistema debe mantener un registro completo del historial de citas de cada paciente, incluyendo fechas, horas, médicos atendientes y motivo de la visita.

\subsection{Gestión de listas de espera}
El sistema debe permitir la gestión de listas de espera para citas canceladas o disponibles, notificando a los pacientes en espera cuando se libera una cita.

\newpage

\section{Requisitos No Funcionales}
\subsection{Usabilidad}
El sistema debe tener una interfaz intuitiva y fácil de usar tanto para los pacientes como para el personal de la clínica, con una navegación sencilla y clara.

\subsection{Rendimiento}
El sistema debe ser capaz de manejar un alto volumen de citas y usuarios simultáneos sin experimentar retrasos o caídas en el rendimiento.

\subsection{Seguridad}
El sistema debe garantizar la confidencialidad y privacidad de la información del paciente, cumpliendo con las regulaciones y estándares de seguridad aplicables.

\subsection{Disponibilidad}
El sistema debe estar disponible las 24 horas del día, los 7 días de la semana, para que los pacientes puedan acceder y programar citas en cualquier momento.

\end{document}
